---
documentclass: article
title: "Contrato de mantenimiento y servicios"
pdftitle: "GuifiBaix: Contrato de mantenimiento y servicios, Alberto Gijón"
pdfauthor: GuifiBaix
header: "GuifiBaix: Contrato de mantenimiento y servicios"
footer:	"GuifiBaix: Contrato de mantenimiento y servicios"
geometry: margin=1in
abstract: "GuifiBaix: Contrato de mantenimiento y servicios"
---

En Sant Joan Despí, a 20 de febrero de 2014


## REUNIDOS

DE UNA  PARTE,
**Alberto Gijón** mayor de edad,
con N.I.F. número **12345678V**,
en adelante, el "CLIENTE",
con domicilio en **Sant Joan Despí**,
**C/Rue Percebe, 13, 4o 3a**,
C.P. **08970**.

DE OTRA  PARTE,
**Ramón Álvarez**,
con N.I.F. número **12345678C**
y en nombre y representación de **AT2, Acció Transversal per la Transformació Social**,
en adelante, la “PROVEEDORA”,
con domicilio en **El Prat de Llobregat**,
**C/Riu Llobregat, 47, Bxos**,
C.P. **08820**
y C.I.F. **G64922131**.

El CLIENTE y la PROVEEDORA, en adelante, podrán ser denominadas,
individualmente, “la Parte” y, conjuntamente, “las Partes”,
reconociéndose mutuamente capacidad jurídica y de obrar suficiente
para la celebración del presente Contrato.

## EXPONEN

PRIMERO: Que el CLIENTE está interesado en la contratación de:

a) Servicio de mantenimiento de los sistemas ofrecidos e instalados (o revisados) por GuifiBaix en los exteriores de la vivienda para la conectividad con la red Guifi.net,
b) Acceso y uso de los servicios proporcionados desde los servidores de GuifiBaix,

SEGUNDO: Que la PROVEEDORA es una entidad especializada en la prestación de servicios de instalación, mantenimiento y conectividad sobre la red Guifi.Net, cumpliendo
con los requerimientos de la red abierta, libre y neutral (Procomún RALN), aportando a la misma sus propios servicios en línea.

TERCERO: Que las Partes están interesadas en celebrar un contrato de Mantenimiento en virtud del cual la PROVEEDORA preste al CLIENTE los servicios de:

a) Mantenimiento de HARDWARE: Mantenimiento de Antenas y cableado, routers y switches, dispositivos de telefonía IP.
b) Mantenimiento de SOFTWARE: Mantenimiento de servicios ofrecidos en la red, y software de antenas y dispositivos de conectividad.
c) Mantenimiento de soporte de RED: Mantenimiento de la conectividad y la seguridad de la misma.


Que las Partes acuerdan celebrar el presente contrato de MANTENIMIENTO Y SERVICIOS,
en adelante, el “CONTRATO”, de acuerdo con las siguientes 


## CLÁUSULAS

### PRIMERA.- OBJETO

En virtud del Contrato la PROVEEDORA se obliga a prestar al CLIENTE los servicios de mantenimiento de hardware, software y de red solicitados, en adelante los “SERVICIOS”, en los términos y condiciones previstos en el Contrato y en todos sus Anexos.



### SEGUNDA.- TÉRMINOS Y CONDICIONES GENERALES Y ESPECÍFICOS DE PRESTACIÓN DE LOS SERVICIOS

2.1. Los Servicios se prestarán en los siguientes términos y condiciones **generales**:

2.1.1. La PROVEEDORA responderá de la calidad del trabajo desarrollado con la diligencia exigible a una empresa experta en la realización de los trabajos objeto del Contrato.

2.1.2. La PROVEEDORA se obliga a gestionar y obtener, a su cargo, todas las licencias, permisos y autorizaciones administrativas que pudieren ser necesarias para la realización de los Servicios. 

2.1.3. La PROVEEDORA se hará cargo de la totalidad de los tributos, cualquiera que sea su naturaleza y carácter, que se devenguen como consecuencia del Contrato, 
así como cualesquiera operaciones físicas y jurídicas que conlleve, salvo el Impuesto sobre el Valor Añadido (IVA) o su equivalente, que la PROVEEDORA repercutirá al CLIENTE.

2.1.4. La PROVEEDORA guardará confidencialidad sobre la información que le facilite el CLIENTE en o para la ejecución del Contrato o que por su propia naturaleza deba ser tratada como tal.
Se excluye de la categoría de información confidencial toda aquella información que sea divulgada por el CLIENTE,
aquella que haya de ser revelada de acuerdo con las leyes o con una resolución judicial o acto de autoridad competente, durante el periodo máximo que estipule la ley.

2.1.5. En el caso de que la prestación de los Servicios suponga la necesidad de acceder a datos de carácter personal,
la PROVEEDORA, como encargado del tratamiento, queda obligado al cumplimiento de la Ley 15/1999,
de 13 de Diciembre, de Protección de Datos de Carácter Personal y del Real Decreto 1720/2007, de 21 de diciembre,
por el que se aprueba el Reglamento de desarrollo de la Ley Orgánica 15/1999 y demás normativa aplicable.

La PROVEEDORA responderá, por tanto, de las infracciones en que pudiera incurrir
en el caso de que destine los datos personales a otra finalidad,
los comunique a un tercero, o en general, los utilice de forma irregular,
así como cuando no adopte las medidas correspondientes para el almacenamiento y custodia de los mismos.
A tal efecto, se obliga a indemnizar al CLIENTE, 
por cualesquiera daños y perjuicios que sufra directamente,
o por toda reclamación, acción o procedimiento,
que traiga su causa de un incumplimiento o cumplimiento defectuoso por parte de la PROVEEDORA
de lo dispuesto tanto en el Contrato
como lo dispuesto en la normativa reguladora de la protección de datos de carácter personal.

A los efectos del artículo 12 de la Ley 15/1999, la PROVEEDORA únicamente tratará los datos de carácter personal a los que tenga acceso conforme a las instrucciones del CLIENTE y no 
los aplicará o utilizará con un fin distinto al objeto del Contrato, ni los comunicará, ni siquiera para su conservación, a otras personas. En el caso de que la PROVEEDORA destine 
los datos a otra finalidad, los comunique o los utilice incumpliendo las estipulaciones del Contrato, será considerado también responsable del tratamiento, respondiendo de las infracciones 
en que hubiera incurrido personalmente. 

La PROVEEDORA deberá adoptar las medidas de índole técnica y organizativas necesarias que garanticen la seguridad de los datos de carácter personal y eviten su alteración, pérdida, tratamiento 
o acceso no autorizado, habida cuenta del estado de la tecnología, la naturaleza de los datos almacenados y los riesgos a que están expuestos, ya provengan de la acción humana o del medio físico 
o natural. A estos efectos la PROVEEDORA deberá aplicar los niveles de seguridad que se establecen en el Real Decreto 1720/2007 de acuerdo a la naturaleza de los datos que trate.

2.1.6. La PROVEEDORA responderá de la corrección y precisión de los documentos que aporte al CLIENTE en ejecución del Contrato y avisará sin dilación al CLIENTE cuando detecte un error para que 
pueda adoptar las medidas y acciones correctoras que estime oportunas.

2.1.7. Las obligaciones establecidas para la PROVEEDORA por la presente cláusula serán también de obligado cumplimiento para sus posibles empleados, colaboradores, tanto externos como internos, 
y subcontratistas, por lo que la PROVEEDORA responderá frente al CLIENTE si tales obligaciones son incumplidas por tales empleados.

2.1.8. El CLIENTE podrá obtener de la PROVEEDORA las especificaciones de su instalación, si lo requiere.

2.2. La PROVEEDORA prestará los Servicios en los siguientes términos y condiciones **específicos**:

a) Mantenimiento de HARDWARE:
	- **Mantenimiento preventivo** que incluye 
		una revisión periódica detallada del correcto funcionamiento
		de los equipos de hardware en todos sus componentes.
	- **Mantenimiento correctivo** que incluye
		la sustitución de los componentes tanto internos como externos que fallaran en los equipos,
		todo ello a cargo del CLIENTE,
		exceptuando los casos de defectos de fábrica o instalación defectuosa.
	- **Mantenimiento evolutivo** que incluye
		la instalación de otros dispositivos adicionales
		para la mejora del rendimiento operativo en general y de seguridad.
		En el momento de la instalación, 
		se determinarán la condiciones de propiedad y mantenimiento de estos nuevos elementos.

b) Mantenimiento de SOFTWARE:
	- **Mantenimiento preventivo** del software instalado que incluye
		la revisión de los parámetros críticos de los equipos y de la red.
	- **Mantenimiento correctivo** que incluye
		la reinstalación o reconfiguración de software en el caso de anomalías en el funcionamiento.
	- **Mantenimiento evolutivo** que incluye
		actualización e instalación de software adicional.
 
c) Mantenimiento de RED: 
	- **Mantenimiento preventivo** que incluye
		la monitorización de los parámetros básicos de la red
		de forma que se garantice su adecuada dimensión.
	- **Mantenimiento correctivo** que incluye
		la reconfiguración del hardware y software de RED
		después de la caída del sistema u otros percances.
	- **Mantenimiento evolutivo** que incluye
		la configuración y optimización para el correcto funcionamiento
		de la comunicación entre los equipos conectados en RED,
		así como todos los periféricos conectados a ella que sean objeto de este contrato.
		
2.2.1. La PROVEEDORA se obliga a prestar los servicios necesarios para el correcto funcionamiento del sistema instalado en el CLIENTE, siempre que no se deban a manipulaciones indebidas. 
Dichos servicios comprenden la realización de cuantas operaciones sean necesarias en los equipos o sistemas incluidos en este contrato.

2.2.2. El CLIENTE facilitará la labor de la PROVEEDORA en todo momento dándole el acceso a los sistemas requeridos. La PROVEEDORA se reserva el acceso físico y telemático a los equipos comunes, a fin de poder
gestionar de forma eficiente y correcta tanto el apartado preventivo, correctivo y/o evolutivo de estos sistemas.

2.2.3. El coste de cualquier componente que la PROVEEDORA tenga sustituir o instalar será por cuenta del CLIENTE, a excepción de defectos de fábrica o mala manipulación por parte del 
PROVEEDOR o terceros vinculados con el mismo. La PROVEEDORA no se responsabiliza de los daños que pudieran producirse por la incorrecta manipulación de los componentes realizados por el CLIENTE o terceros no vinculados.

2.2.4. El CLIENTE se compromete a utilizar los componentes del hardware y del software de acuerdo con las instrucciones y el manual del fabricante.

2.2.5. En ningún caso la PROVEEDORA mantendrá, instalará o configurará sin los debidos permisos administrativos.

2.2.6. La PROVEEDORA ofrecerá un servicio de soporte técnico de 9 a 17 horas todos los días laborables no festivos mediante llamada telefónica al (93-164-0492) o correo electrónico a 
soporte@guifibaix.coop. Las vísperas de festivos nacionales, el horario de atención telefónica será de 9 a 15 horas. Los fines de semana o festivos, el servicio estará disponible a través del 
correo electrónico soporte@guifibaix.coop. Cualquiera incidencia que se produzca, el CLIENTE deberá comunicarla a la PROVEEDORA, a través del teléfono o correo electrónico mencionados, para proceder a su solución. 



### TERCERA.- POLÍTICA DE USO

3.1.
	El CLIENTE es el único responsable de determinar
	si los servicios que constituyen el objeto de este Contrato
	se ajustan a sus necesidades,
	por lo que la PROVEEDORA no garantiza que el servicio de mantenimiento contratado
	se ajuste a las necesidades específicas del CLIENTE.

3.2.
	El CLIENTE se obliga a hacer constar de forma clara, visible y accesible desde sus contenidos,
	sus datos identificativos y como único responsable de los contenidos,
	poniendo un aviso en sus contenidos de la Política de uso de los mismos.

3.3.
	El CLIENTE se compromete a no hacer un uso de la red que altere por sobrecarga o saturación,
	la calidad de servicio ofrecida al resto de usuarios.
	La PROVEEDORA se reserva el derecho de bloquear o modular dichos usos, para garantizar la calidad de servicio.
	La PROVEEDORA se compromete a informar al CLIENTE, si lo pide,
	si un servicio está siendo  modulado o bloqueado e informar de las alternativas.

3.4.
	El CLIENTE se compromete
	a no usar la red con fines delictivos, asi como a no desarrollar actividades que persigan o tengan como consecuencia:

- La congestión de los enlaces de comunicaciones o sistemas informáticos.
- La destrucción o modificación premeditada de la información de otros usuarios.
- La violación de la privacidad e intimidad de otros usuarios.
- El envío de SPAM a través de la red sea cual sea su forma


3.5.
	Tanto CLIENTE como PROVEEDOR, se comprometen a subscribir las condiciones de
	la Licencia de Red de ProComún, anexada a este contrato.



### CUARTA.- RESPONSABILIDAD SOCIAL

4.1. **Sin ánimo de lucro:**
	La PROVEEDORA se compromete a ajustar sus precios y tarifas a la viabilidad económica del proyecto.
	Cualquier superávit se reinvertirá en servicios o en infraestructuras comunes para mejorar la red o
	se dedicará a fondos de reserva.
	Dentro de dicha viabilidad económica, la PROVEEDORA tomará las decisiones
	bajo los criterios sociales que se detallan a continuación.

4.2. **Red libre, abierta y neutral:**
	La PROVEEDORA se compromete cumplir y hacer cumplir,
	en las redes que desplega y mantiene,
	la licencia de Red Neutral y Abierta en Procomún
	anexada al presente contrato.
	La PROVEEDORA se compromete a que los protocolos telemáticos
	usados en dichas redes estén basados estándares abiertos.

4.3. **No atadura a la PROVEEDORA:**
	La PROVEEDORA se compromete a que el CLIENTE pueda acceder, si lo pide,
	a la documentación técnica que describe la arquitectura de la red,
	con el suficiente detalle para que una persona o entidad especializada
	pueda substituir las funciones de la PROVEEDORA.
	Los dispositivos instalados que consten en la factura como pagados quedan en propiedad del CLIENTE.
	El CLIENTE garantizará a la PROVEEDORA el acceso físico y telemático
	a los dispositivos dentro de la Red Libre y Neutral mientras estén conectados a ella
	y la PROVEEDORA lo requiera para que la red siga funcionando,
	y a los dispositivos que estén fuera de dicha red, mientras dure este contrato de mantenimiento,

4.4. **Software libre:**
	La PROVEEDORA se compromete con el CLIENTE a que todos los aplicativos informáticos que desarrollen
	se  publicarán bajo una licencia libre según la definición de la Open Source Initiative.
	También se compromete a desempeñar su actividad con relación al CLIENTE,
	usando exclusivamente software libre, 
	exceptuando las aplicaciones privativas para las que no hubiera alternativa libre viable.
	En este caso, la PROVEEDORA se compromete a poner recursos para viabilizar dicha alternativa.

4.5. **Formatos abiertos:**
	La PROVEEDORA se compromete a comunicarse con el CLIENTE mediante documentos
	en formatos abiertos y libres de royalties
	que no requieran la adquisición de licencias privativas para acceder a ellos.

4.6. **Energías renovables:**
	La PROVEEDORA se compromete con el CLIENTE a que
	los equipos y servidores localizados en sus sedes
	usen exclusivamente energías renovables,
	contratándo la luz de dichas sedes
	con comercializadoras eléctricas que garanticen el origen renovable de la energía comprada.

4.7. **Reducción y tratamiento de residuos e impacto ambiental:**
	La PROVEEDORA se compromete a reducir y reciclar o tratar adecuadamente los residuos generados directamente de su labor.
	La PROVEEDORA se compromete, si se lo pidiera el CLIENTE,
	a dar utilidad en la instalación a los dispositivos electrónicos de los que ya disponga el CLIENTE,
	si funcionalmente fuera viable.
	La PROVEEDORA facilitará la recompra para su utilización en instalaciones nuevas,
	de los dispositivos que, después de instalados, dejasen de usarse en la instalación.
	La PROVEEDORA no esta obligado a dicha recompra ni a hacerlo por el mismo precio que en la compra.

4.8. **Economía local:**
	La PROVEEDORA se compromete con el CLIENTE a potenciar la economía local.
	La PROVEEDORA incluirá la localidad como un criterio social
	más a tener en cuenta en la elección de sus proveedores.
	La PROVEEDORA dará preferencia al pequeño comercio frente a las grandes cadenas comerciales.
	La PROVEEDORA se compromete con el CLIENTE a no deslocalizar trabajo remunerado,
	excluyendo explícitamente el apoyo a proyectos de software libre o hardware libre,
	cuyos equipos por su naturaleza son internacionales.

4.9. **Trabajo digno:**
	La PROVEEDORA se compromete a que los trabajadores que dan el servicio al CLIENTE
	tengan control democrático sobre sus condiciones laborales.
	Los trabajadores serán remunerados en igualdad a igual carga de trabajo,
	sin hacer discriminación por sexo, orientación sexual, raza, discapacidad,
	origen social, religión o ideología.
	La PROVEEDORA se compromete a mantener una horquilla de sueldos por carga de trabajo
	inferior al 1:2, incluyendo en este compromiso las subcontratas.

4.10. **Participación y formación:**
	La PROVEEDORA se compromete a incorporar, mediante procesos democráticos, a los usuarios
	en las decisiones que afecten a la evolución de la red común.
	Para tal fin, la PROVEEDORA fomentará la autoorganizacion democrática del CLIENTE con otros clientes,
	y organizará acciones de formación técnica para que estos puedan tomar decisiones informadas.
	La PROVEEDORA se compromete a organizar acciones de formación 
	de trabajadores y usuarios en el uso de tecnologías libres.

4.11. **Extensión de los compromisos a terceros:**
	La PROVEEDORA incidirá en terceros proveedores
	para que hagan extensivos los compromisos sociales de la PROVEEDORA.
	Con tal fin, la PROVEEDORA se compromete a dar preferencia a aquellos proveedores 
	que cumplan dichos compromisos.
	En casos en los que se tenga constancia o indicio suficiente
	de que una tercera entidad proveedora no está respetando
	los derechos humanos,
	los derechos de los trabajadores,
	los derechos de la infancia
	o esté degradando el medio ambiente
	la PROVEEDORA se compromete a, primero, intentar que cambie la situación,
	y si ello no se produce, buscar terceros proveedores alternativos que sí los respeten.


### QUINTA.- PRECIO Y FACTURACIÓN

Como contraprestación de los servicios prestados por la PROVEEDORA, el CLIENTE abonará
las tarifas adjuntas al presente contrato y que el CLIENTE reconoce conocer y haber recibido
en el momento de su firma.

La PROVEEDORA podrá variar las tarifas para adaptarlas por necesidades del mercado,
previa notificacion al CLIENTE mínimo un mes antes de la emisión de la primera factura
en la que tengan efecto las nuevas tarifas.
Una vez notificadas las nuevas tarifas se entienden aceptadas por el CLIENTE tácitamente.

La PROVEEDORA facturará, trimestralmente y a trimestre vista,
al CLIENTE la cuota pactada incrementada con el IVA.
Los pagos se llevarán a cabo el dia 10
de los meses de enero, abril, julio y octubre
o, en los trimestres incompletos, el dia que comience a proporcionarse el servicio.
La PROVEEDORA facturará los periodos incompletos prorrateando
el número de días naturales efectivos respecto a los dias naturales del trimestre.

En caso de impago de los cargos,
la deuda vencida generará un interés de demora igual al interés legal del dinero
incrementado en dos puntos y dará derecho a la PROVEEDORA a resolver el
Contrato de conformidad con lo dispuesto en las presentes Condiciones.


### SEXTA.- ACUERDO DE NIVEL DE SERVICIO

6.1.
	Todos los Servicios prestados por la PROVEEDORA se realizarán por personal especializado en cada materia.
	El personal de la PROVEEDORA atenderá remotamente o acudirá previsto de todo el material necesario, 
	adecuado y actualizado, para prestar los Servicios.

6.2.
	Las averías o el mal funcionamiento de los Servicios se comunicarán a la PROVEEDORA
	a través de llamada telefónica (93-164-0492) o
	envío de correo electrónico soporte@guifibaix.coop.
	Los horarios de atención, son los especificados en el punto 2.2.6

6.3.
	Los problemas se atenderán en el período máximo establecido según se defina la incidencia.
	El tiempo de resolución final, dependerá de la complejidad de la misma, y del volumen de sistemas y/o usuarios afectados.

6.4.
	Se entiende por *incidencia crítica*:
	las incidencias que, en el marco de la prestación de los Servicios, afectan significativamente al CLIENTE.

6.5.
	Se entiende por *incidencia grave*:
	las incidencias que, en el marco de la prestación de los Servicios, afectan moderadamente al CLIENTE.

6.6.
	Se entiende por *incidencia leve*:
	las incidencias que se limitan a entorpecer la prestación de los Servicios.

6.7.
	El tiempo de atención a dichas incidencias se realizará en los siguientes períodos máximos desde el aviso:

* Incidencia crítica: 30 minutos. (Tiempo máximo de resolución: 1 hora)
* Incidencia grave:   1 hora.
* Incidencia leve:    entre 1 y 2 horas.

6.8.
	El estado de los Servicios se revisará semanalmente por la PROVEEDORA para comprobar su correcto funcionamiento.


### SÉPTIMA.- DURACIÓN DEL CONTRATO

El plazo de duración del presente Contrato es indefinido 
a partir de la fecha referida en el encabezamiento del Contrato
pudiendo el cliente a proceder a su resolución en cualquier momento,
efectuando un preaviso de al menos 30 días de antelación.


### OCTAVA.- SUSPENSION DEL CONTRATO

En caso de que se hubiera producido un impago por parte del CLIENTE,
o en caso de que la PROVEEDORA constate que el CLIENTE incumple alguna de las clausulas de este Contrato,
la PROVEEDORA, después de comunicarlo por los canales especificados en este Contrato,
podría suspender temporalmente la prestación de los servicios que dependan de la PROVEEDORA,
excluyendo las garantías de conectividad con la Red Libre, Abierta y Neutral.

El CLIENTE tiene derecho a solicitar y obtener de la PROVEEDORA la suspensión temporal del servicio
por un periodo determinado que no será menor de un mes, ni superior a tres meses.
El periodo de suspensión no podrá exceder, en ningún caso, de noventa días por año natural.
Durante el tiempo que dure la suspensión temporal se facturará al cliente la mitad del importe mensual,
en concepto de mantenimiento de la red común.

### NOVENA.- MODIFICACIÓN DEL CONTRATO

Las Partes podrán modificar el contrato de mutuo acuerdo y por escrito.


### DÉCIMA.- RESOLUCIÓN DEL CONTRATO

Las Partes podrán resolver el Contrato,
en caso de incumplimiento de las obligaciones establecidas en el mismo.

### ONCEAVA.- NOTIFICACIONES

Las notificaciones que se realicen las Partes deberán realizarse por correo electrónico, teléfono o correo ordinario, a las siguientes direcciones:

	CLIENTE:
		Dirección:	C/Rue Percebe, 13, 4o 3a
					08970 Sant Joan Despí
		Email: 		email@usuario.test
		Teléfono:	93-111-2222

	PROVEEDOR:
		Dirección:	C/Riu Llobregat, 47, Bxos
					08820 El Prat de Llobregat
		Email: 		contacto@guifibaix.coop
		Teléfono:	93-164-0492




### DOCEAVA.- REGIMEN JURÍDICO

El presente contrato tiene carácter mercantil,
no existiendo en ningún caso vínculo laboral alguno entre el CLIENTE
y el personal de la PROVEEDORA que preste concretamente los Servicios.

Toda controversia derivada de este contrato o que guarde relación con él
– incluida cualquier cuestión relativa a su existencia, validez o terminación -
será resuelta mediante arbitraje DE DERECHO, administrado por la
**Asociación Europea de Arbitraje de Madrid (Aeade)**,
de conformidad con su Reglamento de Arbitraje
vigente a la fecha de presentación de la solicitud de arbitraje.
El Tribunal Arbitral que se designe a tal efecto estará compuesto por un único árbitro experto.
La sede del arbitraje será Barcelona.

&nbsp;

Y en prueba de cuanto antecede, las Partes suscriben el Contrato, en dos ejemplares y a un solo efecto, en el lugar y fecha señalados en el encabezamiento.

&nbsp;

\doubleSignature{ EL CLIENTE }{ LA PROVEEDORA }{ Sant Joan Despí }{20 de febrero de 2014}


