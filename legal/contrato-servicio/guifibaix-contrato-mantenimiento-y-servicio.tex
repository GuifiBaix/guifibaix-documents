\documentclass['12pt',spanish,a4paper,]{article}
\usepackage[T1]{fontenc}
\usepackage{lmodern}
\usepackage{amssymb,amsmath}
\usepackage{ifxetex,ifluatex}
\usepackage{fixltx2e} % provides \textsubscript
\usepackage{graphicx}
\usepackage{wrapfig}
\usepackage{ccicons}
% use upquote if available, for straight quotes in verbatim environments
\IfFileExists{upquote.sty}{\usepackage{upquote}}{}
\ifnum 0\ifxetex 1\fi\ifluatex 1\fi=0 % if pdftex
  \usepackage[utf8]{inputenc}
\else % if luatex or xelatex
  \usepackage{fontspec}
  \ifxetex
    \usepackage{xltxtra,xunicode}
  \fi
  \defaultfontfeatures{Mapping=tex-text,Scale=MatchLowercase}
  \newcommand{\euro}{€}
    \setmainfont{'Ubuntu'}
\fi
% use microtype if available
\IfFileExists{microtype.sty}{\usepackage{microtype}}{}
\usepackage{longtable}
\ifxetex
  \usepackage[setpagesize=false, % page size defined by xetex
              unicode=false, % unicode breaks when used with xetex
              xetex]{hyperref}
\else
  \usepackage[unicode=true]{hyperref}
\fi
\hypersetup{breaklinks=true,
            bookmarks=true,
            pdfauthor={},
            pdftitle={Contrato de servicios: GuifiBaix SCCL},
            colorlinks=true,
            urlcolor=blue,
            linkcolor=magenta,
            pdfborder={0 0 0}}
\urlstyle{same}  % don't use monospace font for urls
\setlength{\parindent}{0pt}
\setlength{\parskip}{6pt plus 2pt minus 1pt}
\setlength{\emergencystretch}{3em}  % prevent overfull lines
\setcounter{secnumdepth}{0}
\ifxetex
  \usepackage{polyglossia}
  \setmainlanguage{spanish}
\else
  \usepackage[spanish]{babel}
\fi
\renewcommand*{\familydefault}{\sfdefault}

\author{}
\date{}

\begin{document}


\section{GuifiBaix: Contrato de mantenimiento y
servicios}\label{guifibaix-contrato-de-mantenimiento-y-servicios}

En Sant Joan Despí, a 20 de febrero de 2014

\subsection{REUNIDOS}\label{reunidos}

DE UNA PARTE, \textbf{Alberto Gijón} mayor de edad, con N.I.F. número
\textbf{12345678V}, en adelante, el ``CLIENTE'', domiciliada en
\textbf{Sant Joan Despí}, \textbf{C/Rue Percebe, 13, 4o 3a}, C.P.
\textbf{08970}.

DE OTRA PARTE, \textbf{Ramón Álvarez}, con N.I.F. número
\textbf{12345678C} y en nombre y representación de \textbf{AT2, Acció
Transversal per la Transformació Social}, en adelante, el ``PROVEEDOR'',
domiciliada en \textbf{El Prat de Llobregat}, \textbf{C/Riu Llobregat,
47, Bxos}, C.P. \textbf{08820} y C.I.F. \textbf{G64922131}.

El CLIENTE y el PROVEEDOR, en adelante, podrán ser denominadas,
individualmente, ``la Parte'' y, conjuntamente, ``las Partes'',
reconociéndose mutuamente capacidad jurídica y de obrar suficiente para
la celebración del presente Contrato.

\subsection{EXPONEN}\label{exponen}

PRIMERO: Que el CLIENTE está interesado en la contratación de:

\begin{enumerate}
\def\labelenumi{\alph{enumi})}
\itemsep1pt\parskip0pt\parsep0pt
\item
  Servicio de mantenimiento de los sistemas ofrecidos e instalados (o
  revisados) por GuifiBaix en los exteriores de la vivienda para la
  conectividad con la red Guifi.net,
\item
  Acceso y uso de los servicios proporcionados desde los servidores de
  GuifiBaix,
\end{enumerate}

SEGUNDO: Que el PROVEEDOR es una entidad especializada en la prestación
de servicios de instalación, mantenimiento y conectividad sobre la red
Guifi.Net, cumpliendo con los requerimientos de la red abierta, libre y
neutral (Procomún RALN), aportando a la misma sus propios servicios en
línea.

TERCERO: Que las Partes están interesadas en celebrar un contrato de
Mantenimiento en virtud del cual el PROVEEDOR preste al CLIENTE los
servicios de:

\begin{enumerate}
\def\labelenumi{\alph{enumi})}
\itemsep1pt\parskip0pt\parsep0pt
\item
  Mantenimiento de HARDWARE: Mantenimiento de Antenas y cableado,
  routers y switches, dispositivos de telefonía IP.
\item
  Mantenimiento de SOFTWARE: Mantenimiento de servicios ofrecidos en la
  red, y software de antenas y dispositivos de conectividad.
\item
  Mantenimiento de soporte de RED: Mantenimiento de la conectividad y la
  seguridad de la misma.
\end{enumerate}

Que las Partes acuerdan celebrar el presente contrato de MANTENIMIENTO Y
SERVICIOS, en adelante, el ``CONTRATO'', de acuerdo con las siguientes

\subsection{CLÁUSULAS}\label{cluxe1usulas}

\subsubsection{PRIMERA.- OBJETO}\label{primera.--objeto}

En virtud del Contrato el PROVEEDOR se obliga a prestar al CLIENTE los
servicios de mantenimiento de hardware, software y de red solicitados,
en adelante los ``SERVICIOS'', en los términos y condiciones previstos
en el Contrato y en todos sus Anexos.

\subsubsection{SEGUNDA.- TÉRMINOS Y CONDICIONES GENERALES Y ESPECÍFICOS
DE PRESTACIÓN DE LOS
SERVICIOS}\label{segunda.--tuxe9rminos-y-condiciones-generales-y-especuxedficos-de-prestaciuxf3n-de-los-servicios}

2.1. Los Servicios se prestarán en los siguientes términos y condiciones
\textbf{generales}:

2.1.1. El PROVEEDOR responderá de la calidad del trabajo desarrollado
con la diligencia exigible a una empresa experta en la realización de
los trabajos objeto del Contrato.

2.1.2. El PROVEEDOR se obliga a gestionar y obtener, a su cargo, todas
las licencias, permisos y autorizaciones administrativas que pudieren
ser necesarias para la realización de los Servicios.

2.1.3. El PROVEEDOR se hará cargo de la totalidad de los tributos,
cualquiera que sea su naturaleza y carácter, que se devenguen como
consecuencia del Contrato, así como cualesquiera operaciones físicas y
jurídicas que conlleve, salvo el Impuesto sobre el Valor Añadido (IVA) o
su equivalente, que el PROVEEDOR repercutirá al CLIENTE.

2.1.4. El PROVEEDOR guardará confidencialidad sobre la información que
le facilite el CLIENTE en o para la ejecución del Contrato o que por su
propia naturaleza deba ser tratada como tal. Se excluye de la categoría
de información confidencial toda aquella información que sea divulgada
por el CLIENTE, aquella que haya de ser revelada de acuerdo con las
leyes o con una resolución judicial o acto de autoridad competente,
durante el periodo máximo que estipule la ley.

2.1.5. En el caso de que la prestación de los Servicios suponga la
necesidad de acceder a datos de carácter personal, el PROVEEDOR, como
encargado del tratamiento, queda obligado al cumplimiento de la Ley
15/1999, de 13 de diciembre, de Protección de Datos de Carácter Personal
y del Real Decreto 1720/2007, de 21 de diciembre, por el que se aprueba
el Reglamento de desarrollo de la Ley Orgánica 15/1999 y demás normativa
aplicable.

El PROVEEDOR responderá, por tanto, de las infracciones en que pudiera
incurrir en el caso de que destine los datos personales a otra
finalidad, los comunique a un tercero, o en general, los utilice de
forma irregular, así como cuando no adopte las medidas correspondientes
para el almacenamiento y custodia de los mismos. A tal efecto, se obliga
a indemnizar al CLIENTE, por cualesquiera daños y perjuicios que sufra
directamente, o por toda reclamación, acción o procedimiento, que traiga
su causa de un incumplimiento o cumplimiento defectuoso por parte del
PROVEEDOR de lo dispuesto tanto en el Contrato como lo dispuesto en la
normativa reguladora de la protección de datos de carácter personal.

A los efectos del artículo 12 de la Ley 15/1999, el PROVEEDOR únicamente
tratará los datos de carácter personal a los que tenga acceso conforme a
las instrucciones del CLIENTE y no los aplicará o utilizará con un fin
distinto al objeto del Contrato, ni los comunicará, ni siquiera para su
conservación, a otras personas. En el caso de que el PROVEEDOR destine
los datos a otra finalidad, los comunique o los utilice incumpliendo las
estipulaciones del Contrato, será considerado también responsable del
tratamiento, respondiendo de las infracciones en que hubiera incurrido
personalmente.

El PROVEEDOR deberá adoptar las medidas de índole técnica y
organizativas necesarias que garanticen la seguridad de los datos de
carácter personal y eviten su alteración, pérdida, tratamiento o acceso
no autorizado, habida cuenta del estado de la tecnología, la naturaleza
de los datos almacenados y los riesgos a que están expuestos, ya
provengan de la acción humana o del medio físico o natural. A estos
efectos el PROVEEDOR deberá aplicar los niveles de seguridad que se
establecen en el Real Decreto 1720/2007 de acuerdo a la naturaleza de
los datos que trate.

2.1.6. El PROVEEDOR responderá de la corrección y precisión de los
documentos que aporte al CLIENTE en ejecución del Contrato y avisará sin
dilación al CLIENTE cuando detecte un error para que pueda adoptar las
medidas y acciones correctoras que estime oportunas.

2.1.7. Las obligaciones establecidas para el PROVEEDOR por la presente
cláusula serán también de obligado cumplimiento para sus posibles
empleados, colaboradores, tanto externos como internos, y
subcontratistas, por lo que el PROVEEDOR responderá frente al CLIENTE si
tales obligaciones son incumplidas por tales empleados.

2.2. El PROVEEDOR prestará los Servicios en los siguientes términos y
condiciones \textbf{específicos}:

\begin{enumerate}
\def\labelenumi{\alph{enumi})}
\item
  Mantenimiento de HARDWARE:

  \begin{itemize}
  \itemsep1pt\parskip0pt\parsep0pt
  \item
    Mantenimiento preventivo que incluye una revisión periódica
    detallada del correcto funcionamiento de los equipos de hardware en
    todos sus componentes.
  \item
    Mantenimiento correctivo que incluye la sustitución de los
    componentes tanto internos como externos que fallaran en los
    equipos, todo ello a cargo del CLIENTE, exceptuando los casos de
    defectos de fábrica o instalación defectuosa.
  \item
    Mantenimiento evolutivo que incluye la instalación de otros
    dispositivos adicionales para la mejora del rendimiento operativo en
    general y de seguridad. En el momento de la instalación, se
    determinarán la condiciones de propiedad y mantenimiento de estos
    nuevos elementos.
  \end{itemize}
\item
  Mantenimiento de SOFTWARE:

  \begin{itemize}
  \itemsep1pt\parskip0pt\parsep0pt
  \item
    Mantenimiento preventivo del software instalado que incluye la
    revisión de los parámetros críticos de los equipos y de la red.
  \item
    Mantenimiento correctivo que incluye la reinstalación o
    reconfiguración de software en el caso de anomalías en el
    funcionamiento.
  \item
    Mantenimiento evolutivo que incluye actualización e instalación de
    software adicional.
  \end{itemize}
\item
  Mantenimiento de RED:

  \begin{itemize}
  \itemsep1pt\parskip0pt\parsep0pt
  \item
    Mantenimiento preventivo que incluya la monitorización de los
    parámetros básicos de la red de forma que se garantice su adecuada
    dimensión.
  \item
    Mantenimiento correctivo que incluye la reconfiguración del hardware
    y software de RED después de la caída del sistema u otros percances.
  \item
    Mantenimiento evolutivo que incluye la configuración y optimización
    para el correcto funcionamiento de la comunicación entre los equipos
    conectados en RED, así como todos los periféricos conectados a ella.
  \end{itemize}
\end{enumerate}

2.2.1. El PROVEEDOR se obliga a prestar los servicios necesarios para el
correcto funcionamiento del sistema instalado en el CLIENTE, siempre que
no se deban a manipulaciones indebidas. Dichos servicios comprenden la
realización de cuantas operaciones sean necesarias en los equipos o
sistemas incluidos en este contrato.

2.2.2. El CLIENTE facilitará la labor del PROVEEDOR en todo momento
dandole el acceso a los sistemas requeridos.

2.2.3. El coste de cualquier componente que el PROVEEDOR tenga sustituir
o instalar será por cuenta del CLIENTE, a excepción de defectos de
fábrica o mala manipulación por parte del PROVEEDOR o terceros
vinculados con el mismo. El PROVEEDOR no se responsabiliza de los daños
que pudieran producirse por la incorrecta manipulación de los
componentes realizados por el CLIENTE o terceros no vinculados.

2.2.4. El CLIENTE se compromete a utilizar los componentes del hardware
y del software de acuerdo con las instrucciones y el manual del
fabricante.

2.2.5. En ningún caso el PROVEEDOR mantendrá, instalará o configurará
sin los debidos permisos.

--TODO-- 2.2.6. El PROVEEDOR ofrecerá un servicio de asistencia técnica
y ``Help Desk'' de {[}\ldots{}{]} a {[}\ldots{}{]} entre las (\ldots{})
{[}indicar horario en el que se deben realizar estos servicios e incluir
cuando proceda fines de semana y festivos{]} donde se notificarán las
incidencias que se produzcan. Cualquiera incidencia que se produzca el
CLIENTE se lo comunicará al PROVEEDOR, a través del teléfono de
asistencia técnica, vía fax o correo electrónico, para proceder a su
solución. El PROVEEDOR proporcionará un informe mensual de las
incidencias, donde se indicará el tiempo que ha llevado su solución. Los
plazos de solución de las incidencias constan en la cláusula 6.4 de este
contrato. --TODO--

\subsubsection{TERCERA.- POLÍTICA DE
USO}\label{tercera.--poluxedtica-de-uso}

3.1. El CLIENTE es el único responsable de determinar si los servicios
que constituyen el objeto de este Contrato se ajustan a sus necesidades,
por lo que el PROVEEDOR no garantiza que el servicio de mantenimiento
contratado se ajuste a las necesidades específicas del CLIENTE.

3.2. El CLIENTE se obliga a hacer constar de forma clara, visible y
accesible desde sus contenidos, sus datos identificativos y como único
responsable de los contenidos, poniendo un aviso en sus contenidos de la
Política de uso.

\subsubsection{CUARTA.- RESPONSABILIDAD
SOCIAL}\label{cuarta.--responsabilidad-social}

Lo que DAVID diga\ldots{}. -- TODO --

\subsubsection{QUINTA.- PRECIO Y
FACTURACIÓN.-}\label{quinta.--precio-y-facturaciuxf3n.-}

4.1. El precio del Contrato es de (\ldots{}) {[}indicar el precio de
cada servicio o el total{]} IVA excluido.

4.2. {[}Como debe el cliente pagar la cuota???{]} El pago de las
facturas se realizará, tras la aceptación de los trabajos por el
CLIENTE, mediante transferencia bancaria a los 30 días de la fecha de
recepción de la factura a la siguiente cuenta corriente titularidad del
PROVEEDOR: (\ldots{}) {[}indicar nº de cuenta{]}.

\subsubsection{SEXTA.- DURACIÓN DEL
CONTRATO}\label{sexta.--duraciuxf3n-del-contrato}

El plazo de duración del presente Contrato es indefinido a partir de la
fecha referida en el encabezamiento del Contrato.

\subsubsection{SÉPTIMA.- SUSPENSION DEL
CONTRATO}\label{suxe9ptima.--suspension-del-contrato}

El CLEINTE no paga!!!! -- TODO --

\subsubsection{OCTAVA.- MODIFICACIÓN DEL
CONTRATO}\label{octava.--modificaciuxf3n-del-contrato}

Las Partes podrán modificar el contrato de mutuo acuerdo y por escrito.

\subsubsection{NOVENA.- RESOLUCIÓN DEL
CONTRATO}\label{novena.--resoluciuxf3n-del-contrato}

Las Partes podrán resolver el Contrato, en caso de incumplimiento de las
obligaciones establecidas en el mismo.

\subsubsection{DÉCIMA.- ACUERDO DE NIVEL DE
SERVICIO}\label{duxe9cima.--acuerdo-de-nivel-de-servicio}

10.1. Todos los Servicios prestados por el PROVEEDOR se realizarán por
personal especializado en cada materia. El personal del PROVEEDOR
acudirá previsto de todo el material necesario, adecuado y actualizado,
para prestar los Servicios.

10.2. Las averías o el mal funcionamiento de los Servicios se
comunicarán al PROVEEDOR en su domicilio a través de llamada telefónica
(93-164-0492) o envío de correo electrónico incidencias@guifibaix.coop.

10.3. Los problemas se resolverán en un período máximo de (\ldots{})
{[}establecer distintos plazos a tendiendo a la gravedad de la
incidencia, leve, grave, crítica. las penalizaciones se deben definir
también en función de la gravedad de la incidencia{]}

10.4. Se entiende por \emph{incidencia crítica}: las incidencias que, en
el marco de la prestación de los Servicios, afectan significativamente
al CLIENTE. {[}si se pueden establecer parámetros objetivos mejor{]}

Se entiende por \emph{incidencia grave}: las incidencias que, en el
marco de la prestación de los Servicios, afectan moderadamente al
CLIENTE. {[}si se pueden establecer parámetros objetivos mejor{]} Se
entiende por \emph{incidencia leve}: las incidencias que se limitan a
entorpecer la prestación de los Servicios. {[}si se pueden establecer
parámetros objetivos mejor{]} La reparación se realizará en los
siguientes períodos máximos desde el aviso:

\begin{itemize}
\itemsep1pt\parskip0pt\parsep0pt
\item
  Incidencia crítica (\ldots{})
\item
  Incidencia grave (\ldots{})
\item
  Incidencia leve (\ldots{}) {[}indicar períodos para los SLA
  aproximados{]}
\end{itemize}

10.5. El estado de los Servicios se revisará (\ldots{})
{[}semanalment/mensualmente/bimensualmente/trimestralmente{]} por el
CLIENTE y el PROVEEDOR para comprobar su buen funcionamiento.

\subsubsection{ONCEAVA.- NOTIFICACIONES}\label{onceava.--notificaciones}

Las notificaciones que se realicen las Partes deberán realizarse por
correo electrónico, teléfono o correo ordinario, a las siguientes
direcciones:

CLIENTE:

\begin{verbatim}
Dirección:  C/Rue Percebe, 13, 4o 3a
            08970 El Prat de Llobregat
Email:      email@usuario.test
Teléfono:   93-111-2222
\end{verbatim}

PROVEEDOR:

\begin{verbatim}
Dirección:  C/Riu Llobregat, 47, Bxos
            08820 El Prat de Llobregat
Email:      contacto@guifibaix.coop
Teléfono:   93-164-0492
\end{verbatim}

\subsubsection{DOCEAVA.- REGIMEN
JURÍDICO}\label{doceava.--regimen-juruxeddico}

El presente contrato tiene carácter mercantil, no existiendo en ningún
caso vínculo laboral alguno entre el CLIENTE y el personal del PROVEEDOR
que preste concretamente los Servicios.

Toda controversia derivada de este contrato o que guarde relación con él
--incluida cualquier cuestión relativa a su existencia, validez o
terminación- será resuelta mediante arbitraje DE DERECHO, administrado
por la \textbf{Asociación Europea de Arbitraje de Madrid (Aeade)}, de
conformidad con su Reglamento de Arbitraje vigente a la fecha de
presentación de la solicitud de arbitraje. El Tribunal Arbitral que se
designe a tal efecto estará compuesto por un único árbitro experto. La
sede del arbitraje será Barcelona.

~

Y en prueba de cuanto antecede, las Partes suscriben el Contrato, en dos
ejemplares y a un solo efecto, en el lugar y fecha señalados en el
encabezamiento.

~

\begin{longtable}[c]{@{}llllllllllllll@{}}
\toprule\addlinespace
POR EL CLIENTE & & & & & & & POR EL PROVEEDOR & & & & & &
\\\addlinespace
~ & ~ & ~ & ~ & ~ & ~ & ~ & ~ & ~ & ~ & ~ & ~ & ~ & ~
\\\addlinespace
~ & ~ & ~ & ~ & ~ & ~ & ~ & ~ & ~ & ~ & ~ & ~ & ~ & ~
\\\addlinespace
~ & ~ & ~ & ~ & ~ & ~ & ~ & ~ & ~ & ~ & ~ & ~ & ~ & ~
\\\addlinespace
~ & ~ & ~ & ~ & ~ & ~ & ~ & ~ & ~ & ~ & ~ & ~ & ~ & ~
\\\addlinespace
~ & ~ & ~ & ~ & ~ & ~ & ~ & ~ & ~ & ~ & ~ & ~ & ~ & ~
\\\addlinespace
~ & ~ & ~ & ~ & ~ & ~ & ~ & ~ & ~ & ~ & ~ & ~ & ~ & ~
\\\addlinespace
~ & ~ & ~ & ~ & ~ & ~ & ~ & ~ & ~ & ~ & ~ & ~ & ~ & ~
\\\addlinespace
Fdo.: & & & & & & & Fdo.: & & & & & &
\\\addlinespace
\bottomrule
\end{longtable}

\section{Correcciones (David)}\label{correcciones-david}

\begin{itemize}
\itemsep1pt\parskip0pt\parsep0pt
\item
  GuifiBaix se reserva el acceso físico y telemático a los equipos
  comunes.
\item
  Cuando se habla de `su(s) sistema(s) informatico(s)' se puede entender
  que pasamos a mantenerle la red interna de casa, lo cual es falso
\item
  Garantias equipos
\item
  Exclusion por daños no debidos a nuestra manipulación
\item
  El usuario puede obtener de GuifiBaix las especificaciones de su
  instalación.
\item
  Se usaran estándares abiertos cuando sea viable.
\item
  El software usado sera basado en software libre cuando sea viable.
\item
  Hay transferencia de datos a NubipTel, por ejemplo
\end{itemize}

\end{document}
